\documentclass[]{article}
\setlength{\parindent}{0in}
\usepackage{amsfonts}
\usepackage{amssymb}
\begin{document}

\title{\bf{CS 241: Principles of Sequential Programming}}
\date{Winter 2013, University of Waterloo}
\author{Chris Thomson}
\maketitle
\newpage

\section{Introduction (January 7, 2013)}
	\textbf{Abstraction} is the process of removing or hiding irrelevant details. Everything is just a sequence of bits (binary digits). There are two possible values for a bit, and those values can have arbitrary labels such as:
	\begin{itemize}
		\item Up / down.
		\item Yes / no.
		\item 1 / 0.
		\item On / off.
		\item Pass / fail.
	\end{itemize}
	
	Let's say we have four projector screens, each representing a bit of up/down, depending on if the screen has been pulled down or left up (ignoring states between up and down). These screens are up or down independently. There are sixteen possible combinations: \\
	\begin{center}
		\begin{tabular}{cccc}
			\underline{Screen 1} & \underline{Screen 2} &\underline{Screen 3} & \underline{Screen 4} \\
			Up (1) & Down (0) & Up (1) & Down (0) \\
			Down (0) & Down (0) & Down (0) & Up (1) \\
			\vdots & \vdots & \vdots & \vdots \\
		\end{tabular}	
	\end{center}

	Note that there are sixteen combinations because $k = 4$, and there are always $2^k$ combinations since there are two possible values for each of $k$ screens.
	\\ \\
	Let's think about the sequence $1010$. This sequence of bits has a different interpretation when following different conventions.
	\begin{itemize}
		\item \textbf{Unsigned, little-endian}: $(1 \times 2^0) + (0 \times 2^1) + (1 \times 2^2) + (0 \times 2^3) = 1 + 4 = 5$.
		\item \textbf{Unsigned, big-endian}: $(0 \times 2^0) + (1 \times 2^1) + (0 \times 2^2) + (1 \times 2^3) = 2 + 8 = 10$.
		\item \textbf{Two's complement, little-endian}: $5 - 16 = -10$.
		\item \textbf{Two's complement, big-endian}: $10 - 16 = -6$.
		\item \textbf{Computer terminal}: LF (line feed). \\
	\end{itemize}

	Note that a two's complement number $n$ will satisfy $-2^{k-1} \le n < 2^{k-1}$. \\

	\textbf{ASCII} is a set of meanings for 7-bit sequences.
	
	\begin{center}
		\begin{tabular}{cc}
			\underline{Bits} & \underline{ASCII Interpretation} \\
			$0001010$ & LF (line feed) \\
			$1000111$ & G \\
			$1100111$ & g \\
			$0111000$ & 8
		\end{tabular}
	\end{center}
	
	In the latter case, $0111000$ represents the character '$8$', not the unsigned big- or little-endian number $8$.
	\\ \\
	ASCII was invented to communicate text. ASCII can represent characters such as A-Z, a-z, 0-9, and control characters like ();!. Since ASCII uses 7 bits, $2^7 = 128$ characters can be represented with ASCII. As a consequence of that, ASCII is basically only for Roman, unaccented characters, although many people have created their own variations of ASCII with different characters.
	\\ \\
	\textbf{Unicode} was created to represent more characters. Unicode is represented as a 32-bit binary number, although representing it using 20 bits would also be sufficient. The ASCII characters are the first 128, followed by additional symbols.
	\\ \\
	A 16-bit representation of Unicode is called \textbf{UTF-16}. However, there's a problem: we have \emph{many} symbols ($> 1M$) but only $2^16 = 65,536$ possibilities to represent them. Common characters are represented directly, and there is also a 'see attachment' bit for handling the many other symbols that didn't make the cut to be part of the $65,536$. Similarly, there is an 8-bit representation of Unicode called \textbf{UTF-8}, with the ASCII characters followed by additional characters and a 'see attachment' bit.
	\\ \\
	The bits themselves do not have meaning. Meaning is in your head -- everything is up for interpretation.
\end{document}