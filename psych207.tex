\documentclass[]{article}
\usepackage[margin = 1.5in]{geometry}
\setlength{\parindent}{0in}
\usepackage{amsfonts}
\usepackage{amssymb}
\usepackage{hyperref}
\usepackage[T1]{fontenc}
\usepackage{ae,aecompl}

\setlength{\marginparwidth}{1.5in}
\newcommand{\lecture}[1]{\marginpar{{\footnotesize $\leftarrow$ \underline{#1}}}}

\begin{document}
	
	\title{\bf{PSYCH 207: Cognitive Processes}}
	\date{Winter 2013, University of Waterloo \\ \center Notes written from Jonathan Fugelsang's lectures.}
	\author{Chris Thomson}
	\maketitle
	\newpage

	\section{Introduction \& Course Structure} \lecture{January 8, 2013}
		\subsection{Course Structure}
			The grading scheme is four in-class non-cumulative multiple-choice exams, equally weighted. There is also a 4\% bonus for research participation through SONA. You should get the textbook.
			\\ \\
			See the course syllabus for more information -- it's available on \href{https://learn.uwaterloo.ca/}{Waterloo LEARN}.
		
		\subsection{Introduction to Cognitive Processes}
			\begin{quote}
				``Cognitive psychology refers to all processes by which the sensory input is transformed, reduced, elaborated, stored, recovered, and used." \textendash{}  Neisser, 1967
			\end{quote}
			\textbf{Cognitive psychology} involves perception, attention, memory, knowledge, reasoning, and decision making.
			\\ \\
			\textbf{Cognitive processes} are everything that goes on in our mind that affects our environment. Many of these processes are completely unconscious.
			\\ \\
			Conscious experience is an \underline{active reconstructive process}. The external world and our internal representation of that world is \emph{not} an exact match. Our brain ends up filling in many gaps, making many assumptions.
			\\ \\
			Our brain cannot decontextualize the world.
			
	\section{Historical Overview \& Approaches} \lecture{January 10, 2013}
		\underline{Attention} (notice \emph{something}) $\to$ \underline{Perception} (perceive that \emph{something}) $\to$ \underline{Pattern Recognition} (recognize what that \emph{something} is) $\to$ \underline{Memory} (recall previously-known attributes about the \emph{something}).
		\\ \\
		Our cognitive apparatus is ultimately an efficient simplification process.
		
		\subsection{Antecedent Philosophies and Traditions}
			Many researchers take very strong views on empiricism vs. nativism, however reality is most likely somewhere between the two. The debate for structuralism vs. functionalism is similar.
		
			\subsubsection{Empiricism}
				\begin{itemize}
					\item Locke, Hume, and Stuart Hill.
					\item Emphasis is on experience and \underline{learning}.
					\item Key is the \underline{association} between experiences.
					\item This is observational learning -- the nurture side of the nature vs. nurture argument.
				\end{itemize}
				
			\subsubsection{Nativism}
				\begin{itemize}
					\item Plato, Descartes, and Kant.
					\item Emphasis is on that which is \underline{innate}.
					\item Innate causal mechanisms.
					\item This is the nature side of the nature vs. nurture argument.
				\end{itemize}
				
			\subsubsection{Structuralism}
				\begin{itemize}
					\item Wundt and Baldwin.
					\item The focus is on the elemental components of mind.
					\item Very reductive -- it's about stripping out context to understand the very basic elements.
					
					\item \underline{Introspection} (method)
						\begin{itemize}
							\item Report on the basic elements of consciousness.
							\item Not internal perception, but \underline{experimental self observation}.
							\item Must be done in a lab under controlled conditions.
							\item Basic elements of the conscious experience include processes like identifying colors.
						\end{itemize}
				\end{itemize}
			
			\subsubsection{Functionalism}
				\begin{itemize}
					\item William James.
					\item Regarded the mission of psychology to be the explanation of our experience.
					\item Key question: why does the mind work as it does?
					\item The function of our mind is more important than its content.
					\item \underline{Introspection in natural settings} (method)
						\begin{itemize}
							\item Must study the whole organism in real-life situations.
							\item Must get out of the lab to conduct functionalist research.
						\end{itemize}
				\end{itemize}
			
			\subsubsection{Behaviorism}
				\begin{itemize}
					\item Watson and Skinner.
					\item Started in the 30s and was the dominate focus of academic psychology until the 60s.
					\item Originally evolved as a reaction to the lack of progress provided by introspection.
					\item A behaviorist sees psychology as an objective, experimental branch of science. Psychology's goal is the prediction and control of behaviour. Therefore, they make behaviour (not consciousness) the focus of their research.
					\item Focus is on the relation between input and output, but the steps in between (which make up cognitive psychology) do not matter to behaviorists.
				\end{itemize}
				
			\subsubsection{Gestalt Psychology}
				\begin{itemize}
					\item Wertheimer, Koffka, and Kohler.
					\item Focus is on the holistic aspects of conscious experiences.
					\item Key question: what are the rules by which we parse the world into wholes?
					\item \underline{Introspection} (method)
						\begin{itemize}
							\item Experience is simply described, never analyzed.
						\end{itemize}
					\item A unified whole is often different than the sum of its parts. How do we impose structure on what's already out there? For example: 8 line segments in groups of 2 are interpreted differently than the 8 lines being all scrambled together in a seemingly random way.
					\item How does the mind simplify the world to focus our attention on things/objects that matter?
					\item We need to study phenomena in their entirety, since a unified whole is different than the sum of its parts.
				\end{itemize}
			
			\subsubsection{Individual Differences}
				\begin{itemize}
					\item Sir Francis Galton.
					\item Intelligence, morals, and personality are innate.
					\item Mental imagery was studied in both a lab and in natural settings. The vividness of mental imagery differs from person to person.
					\item Galton invented the process of using questionnaires to assess abilities. This process has been used by cognitive psychologists ever since.
				\end{itemize}
			
		\subsection{The Cognitive Revolution}
			\begin{itemize}
				\item The speed of information publishing, sharing, and retrieval has become very fast.
				\item We're now running into the cognitive speed limit as our limiting factor, whereas before communication channels (snail mail, travel) slowed down research.
				\item Recent advances in meuroimaging are also a mini-revolution in cognitive psychology.
			\end{itemize}
			
			\subsubsection{Human factors engineering presented new problems}
				\begin{itemize}
					\item A machine should be designed for human use -- for use in the most efficient way possible. Knowledge of human cognition is required in order to increase efficiency.
					\item We have to think about the $7 \pm 2$ information limitation of the human mind, and how to get around the limit.
					\item NASA hires cognitive psychologists to study how the human mind operates in extreme conditions. Cognitive psychologists develop the user interfaces that astronauts use.
				\end{itemize}
			
			\subsubsection{Behaviorism failed to adequately explain language}
				\begin{itemize}
					\item Skinner in 1957 (behaviorism): children learn language by imitation and reinforcement.
					\item Chomsky in 1959 questioned Skinner's explanation of language.
						\begin{itemize}
							\item Children often say sentences they have never heard before, such as ``I hate you mommy.'' (Not imitation.)
							\item Children often use incorrect grammar, such as ``The boy hitted the ball'', despite a lack of reinforcement.
						\end{itemize}
				\end{itemize}
				
			\subsubsection{Localization of functions in the brain forced discussion of mind}
				\begin{itemize}
					\item Donald Hebb stated that some functions, like perception, are based on cell assemblies (collections of neurons).
					\item Hubel and Weisel demonstrated the importance of early experiences on the development of the nervous system. Early experiences actually change how some cell assemblies physically develop.
					\item Many things seem to happen without observational learning coming into play.
				\end{itemize}
				
			\subsubsection{Development of computers and artificial intelligence gave a dominant metaphor}
				\begin{itemize}
					\item A computer takes input into short-term memory (RAM), may acceess long-term memory (a hard drive), and returns some output.
					\item The mind may work in a similar way.
					\item Perhaps we introspected and that's why we developed computers the way we did?
				\end{itemize}
				
		\subsection{Paradigms of Cognitive Psychology}
			\begin{itemize}
				\item Emphasis is on serial processing.
				\item Information is stored symbolically.
				\item The mind is an information processing system with systems of interrelated capacities.
				\item All of these attributes are similar to that of typical computer systems.
			\end{itemize}
			
			\subsubsection{Localist models}
				\begin{itemize}
					\item A symbolic concept, such as a letter, word, or meaning, is represented in your mind with a node.
					\item You may have a node for `cat', `dot', or `house' (lexical knowledge). You may also have a node for `provides shelter', `barks', or `has four legs', all of which are boolean attributes (semantic knowledge).
				\end{itemize}
				
			\subsubsection{Connectionism -- Neural network models}
				\begin{itemize}
					\item Parallel processing across a population of neurons.
					\item Multiple neurons are used to represent complex concepts. For example: the representation of a person may have a neuron for their name, a neuron for their profession, a neuron for their cat's name, and so on.
					\item The \underline{pattern of activation} of the neurons represent a symbolic concept.
					\item Semantic knowledge and lexical knowledge for a particular symbolic concept have different activation patterns.
					\item Units in neural networks are connected by weights that are modified by learning (positive weight $\to$ activation, negative weight $\to$ inhibition).
				\end{itemize}
		
		\subsection{Major Assumptions of Approaches}
			The major assumption of these approaches is that research must be done in the lab. This is believed for two key reasons:
			\begin{itemize}
				\item We must uncover the basic processes underlying cognition in order to fully understand it.
				\item Processes are stable across situations, and can only be researched under controlled conditions (such as in a lab).
			\end{itemize}
		
		\subsection{Other Approaches}
			\subsubsection{The Evolutionary Approach}
				\begin{itemize}
					\item Mental processes are subject to natural selection.
					\item Cognition is based off our history, and special processes have developed over time.
				\end{itemize}
				
			\subsubsection{The Ecological Approach}
				\begin{itemize}
					\item Cognitive processes develop with culture and differ depending on the context and situation.
					\item Analyzes how humans behave in context-specific situations. As a result of this approach, natural observation must be used instead of lab research.
					\item People focus on the eyes of others, because they show attention, desires, and more. This behavior might differ depending on the context.
				\end{itemize}
\end{document}
