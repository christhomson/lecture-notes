\documentclass[]{article}
\usepackage[margin = 1.5in]{geometry}
\setlength{\parindent}{0in}
\usepackage{amsfonts}
\usepackage{amssymb}
\usepackage{hyperref}
\usepackage{cleveref}
\usepackage{amsmath}
\usepackage{amsthm}
\usepackage{mathtools}
\usepackage{tikz}
\usepackage{qtree}
\usepackage{float}
\usepackage[lined]{algorithm2e}
\usepackage[T1]{fontenc}
\usepackage{ae,aecompl}
\usepackage{color}
\usepackage{bussproofs}
\usepackage{enumerate}

\DeclarePairedDelimiter{\set}{\lbrace}{\rbrace}

\definecolor{darkish-blue}{RGB}{25,103,185}

\hypersetup{
    colorlinks,
    citecolor=darkish-blue,
    filecolor=darkish-blue,
    linkcolor=darkish-blue,
    urlcolor=darkish-blue
}

\theoremstyle{definition}
\newtheorem*{defn}{Definition}
\newtheorem*{theorem}{Theorem}
\newtheorem*{corollary}{Corollary}
\newtheorem*{aside}{Aside}
\newtheorem{ex}{Example}[section]

\crefname{ex}{Example}{Example}

\setlength{\marginparwidth}{1.5in}
\newcommand{\lecture}[1]{\marginpar{{\footnotesize $\leftarrow$ \underline{#1}}}}

\stepcounter{footnote} % because daggers are cool

\begin{document}
	\let\ref\Cref

	\title{\bf{CS 360: Introduction to the Theory of Computing}}
	\date{Fall 2013, University of Waterloo \\ \center Notes written from Shai Ben-David's lectures.}
	\author{Chris Thomson\thanks{See \href{http://cthomson.ca/notes}{cthomson.ca/notes}.
	\ifdefined\sha % Also, \commitDateTime should be defined.
		Last modified: \commitDateTime{} ({\href{https://github.com/christhomson/lecture-notes/commit/\sha}{\sha}}).
	\fi}}

	\maketitle
	\newpage
	\tableofcontents
	\newpage

	\section{Introduction \& Course Structure} \lecture{September 10, 2013}
    \subsection{Course Structure}
      The grading scheme is 30\% assignments, 30\% midterm, and 40\% final. If your final grade is better than your midterm grade, your final will be worth 70\% and the midterm will be worth 0\%.
      \\ \\
      In terms of textbooks, any textbook relating to ``automata theory'', ``theory of computation'', or ``formal language theory'' is acceptable.
      \\ \\
      The professor's office is in DC 1311, and his office hours are on Tuesdays from 2:00 to 3:00 pm.
      \\ \\
      This course is mathematical. The relevant background required is heavily logic-based, and requires (almost) no programming experience.

    \subsection{Introduction}
      The goal of the course is to develop the theory of computation. We want to understand the mathematics that govern computers. We will explore questions like:
      \begin{itemize}
        \item What are computers?
        \item Which tasks can computers carry out?
        \item Which tasks are easy or hard for computers?
      \end{itemize}

      We make abstractions and use them as common properties that are relevant to many phenomenon. We aren't going to be focussing on any particular language, but instead, we'll focus on common properties.
      \\ \\
      We aim to create abstractions that will allow us to generalize with fundamental properties. These abstractions allow us to form conclusions that are relevant to computing in general, and not just for today's hardware and software.
      \\ \\
      This course has two key goals:
      \begin{enumerate}
        \item To familiarize you with the fundamentals of computer science theory.
        \item To develop the ability to reason formally and abstractly about computing.
      \end{enumerate}

  \section{CS 245 Review}
    We'll start off by reviewing some of the basics that were covered on CS 245, to ensure that everyone's on the same page.
    \\ \\
    Throughout the course, we will follow two main threads concurrently: abstract models of computers, and computing tasks. We'll progress from simple models and simple tasks to complex models and complex tasks, and along the way connections will be made between the two.

    \subsection{Computing Tasks}
      We're going to focus on \textbf{decision problems}. Decisions problems are problems which have an input and produce a binary (yes/no) output. Some examples of decision problems are:
      \begin{itemize}
        \item Input: a number. Decision: is it a prime number?
        \item Input: email message. Decision: is it spam?
        \item Input: a graph. Decision: is it connected?
      \end{itemize}

      We're interested in seeing which of these questions a computer can figure out.

      \subsubsection{Modelling Decision Problems}
        We model decision problems using \textbf{formal languages}. A decision problem has two components:
        \begin{itemize}
          \item A domain set $X$.
          \item A language $L \subseteq X$.
        \end{itemize}

        Our decision problem is then ``for some $x \in X$, is $x \in L$?'' Re-examining our examples from earlier, we would define:
        \begin{itemize}
          \item $X = \mathbb{Z}$ (the set of all integers), and $L = $ the set of all prime numbers.
          \item $X = $ the set of all email messages, and $L = $ the set of all spam email messages.
          \item $X = $ the set of all graphs, and $L = $ the set of all connected graphs.
        \end{itemize}

        For further concreteness, we will fix some finite set $\Sigma$, which we will call the \textbf{alphabet}. Some examples of alphabets are $\set{a, b}$, $\set{0, 1}$, and $\set{0, 1, 2, 3}$.
        \\ \\
        $\Sigma^\star$ represents the set of all finite strings over the alphabet $\Sigma$. In many cases, we define $X = \Sigma^\star$. For example, if $\Sigma = \set{0, 1}$ then $\Sigma^\star = \set{ \epsilon, 0, 1, 00, 11, 01, 10, \ldots }$. Note that $\Sigma^\star$ is an infinite set, but every member of $\Sigma^\star$ is finite.
        \\ \\
        We will define just one operation for strings, for the time being: \textbf{concatenation}. Given two strings, $\sigma$ and $\eta$, we get $\sigma \eta$. That is,
          \AxiomC{$\sigma$}
          \AxiomC{$\eta$}
          \BinaryInfC{$\sigma \eta$}
          \DisplayProof . For example, if $\sigma = 01$ and $\eta = 111$, then $\sigma \eta$ is 01111.
        \\ \\
        \textbf{Languages} are subsets of $\Sigma^\star$ (that is, a language is a collection of strings). Some examples of languages include:
        \begin{itemize}
          \item $L = \emptyset$. This is the \textbf{empty language}, which no strings belong to. Note that $|L| = 0$. This is analogous to an empty fridge.
          \item $L = \set{ \epsilon }$. This is a non-empty language that contains one element ($|L| = 1$), which just happens to be $\epsilon$. This is analogous to having a fridge that contains only an empty can.
          \item $L = \set{\epsilon, 0, 01, 11111}$.
          \item $L = \Sigma^\star$.
        \end{itemize}

        Compilers use languages to check their input (the code of a program) to see if the program is valid (if the code is in the language).
        \\ \\
        We will discuss tasks that help us determine if a particular string belongs to a particular language or not.

      \subsubsection{Formal Tools for Defining Sets}
        We want languages to model every computational task, some of which are complex. We need several different methods for formally defining sets, since all languages are sets.

        \begin{enumerate}
          \item \textbf{List all members of the set}. This method is very precise and concrete, but it makes look-ups difficult ($O(n)$) and it fails if the set is infinitely large.
          \item \textbf{Common property that characterizes the set}. For example, consider the set of all even numbers. This set is not definable by a list because it's an infinite set. However, there is a common property that's shared between all members of the set: if you divide by two and examine the remainder, every member in the set will have a remainder of zero.
          \item \textbf{Induction}. Methods (1) and (2) don't offer us a way to define a set of all your blood relatives, since there's no common property. Instead, we could use induction to define that set.
          \\ \\
          Defining a set with induction requires three components:
          \begin{itemize}
            \item A domain set $X$ (sometimes omitted if it is obvious).
            \item A finite core set $A$ (as in ``atoms''), such as $\set{ \text{me} }$.
            \item A finite set of operations $P$, such as $\set{ \text{``son of''}, \text{``father of''}, \text{``mother of''}}$. This set is a set of functions from $X \to X$.
          \end{itemize}

          Then, we would say that the \textbf{inductive set} $I(A, P)$, the set of blood relatives, is defined as all people that can be reached from the core set by applying a finite sequence of operations.
          \\ \\
          More formally, we say that $I(A, P)$ is the inductive set defined by core $A$ and operations $P$, and that $I(A, P)$ is the smallest set which:
          \begin{itemize}
            \item Contains all members of $A$, and
            \item Is closed under the operations in $P$.
          \end{itemize}

          Let's say we define the core set as $\set{2}$, and the set of operations as $\set{ \text{``add 2''}, \text{``subtract 2''}}$. The defined set is the set of all even numbers, both positive and negative.
          \\ \\
          Or, let's say we want to define the set of all algebraic expressions. Our core set would be $\set{x, y, z, 1, 2, 3, \ldots, a, b, c, \ldots}$. Our operations would be:
          \AxiomC{$\sigma$} \AxiomC{$\eta$} \BinaryInfC{$(\sigma + \eta)$} \DisplayProof,\AxiomC{$\sigma$} \AxiomC{$\eta$} \BinaryInfC{$(\sigma \cdot \eta)$} \DisplayProof.
          \\ \\
          Given a set (language) defined inductively, how can we tell if some $x \in X$ is in the language or not? We must provide a particular sequence of operations necessary to produce that $x$.

          \begin{ex}
            \label{strConcatenationExample}
            Let $X = \set{a, b}^\star$, $A = \set{a}$, and $P = $ \bigg\{ \AxiomC{X} \UnaryInfC{aX} \DisplayProof, \AxiomC{X} \UnaryInfC{Xa} \DisplayProof, \AxiomC{X} \AxiomC{Y} \BinaryInfC{bXY} \DisplayProof, \AxiomC{X} \AxiomC{Y} \BinaryInfC{XbY} \DisplayProof, \AxiomC{X} \AxiomC{Y} \BinaryInfC{XYb} \DisplayProof \bigg\}.
            \\ \\
            A typical question would be: is $abbaa \stackrel{?}{\in} I(A, P)$? The answer: yes! However, we must show how to generate $abbaa$ using the operations in $P$, in a finite number of steps.

            \begin{enumerate}[1.]
            \item $a$, belongs to the core $A$.
            \item $baa$, after applying $P_3(a, a)$.
            \item $abbaa$, after applying $P_4(a, baa)$.
            \end{enumerate}

            Note that the first step must always be an element in the core set.
            \\ \\
            Is $\epsilon \in I(A, P)$? No. We could use simple reasoning to show that $\epsilon \not \in I(A, P)$ (each function in $P$ increases the length). However, in general, it is more difficult to prove non-membership of an element than it is to prove membership of another element.
          \end{ex}

          A \textbf{certificate} (or \textbf{generating sequence}) for $x$ is a sequence of elements of $X, \theta_1, \theta_2, \theta_3, \ldots, \theta_n$ such that:
          \begin{itemize}
            \item $\theta_n = x$, and
            \item Every $\theta_i$ in the sequence is either an element of the core set $A$ or is the result of applying an operation from $P$ to a $\theta_j, \theta_l$ that appeared earlier in the sequence ($j < i$ and $l < i$). \lecture{September 12, 2013}
          \end{itemize}
        \end{enumerate}

        Our definition of $I(A, P)$ is still somewhat informal, however. We haven't formally defined what it means for a subset to be the smallest subset, and we also haven't formally defined what it means for a set to be ``closed'' under a set of operations.

        \subsubsection{More on Inductive Sets and Structural Induction}
        More formally, we say that a set $B$ is \textbf{closed under the operations of $\boldsymbol P$} if for every $f \in P$ and every $x, y \in B$, $f(x, y) \in B$. For example, the set of even numbers is closed under the operation +.
        \\ \\
        Now, let's redefine $I(A, P)$ in a more formal way:
        $$
          I(A, P) = \bigcap \left\{ B : A \subseteq B \text{ and } B \text{ is closed under } P \right\}
        $$

        Note that $I(A, P)$ is a unique set which is the intersection over the \emph{entire} collection of sets.

        \begin{ex}
          Let $X = \mathbb{N}, A = \set{10}, P = \set{+}$.
          \\ \\
          $I(A, P)$ is defined as the intersection of all sets $B$ such that $ A \subseteq B $ and $B$ is closed under $P$ (that is, $B$ is closed under $+$). So, we have:
          \begin{itemize}
            \item $B_1 = \set{\text{all even numbers}}$
            \item $B_2 = \set{\text{all numbers divisible by 5}}$
            \item $B_3 = \set{\text{all numbers divisible by 10}}$
          \end{itemize}

          In this case, $I(A, P) = \set{10, 20, \ldots}$, since $I(A, P)$ must be the minimal set among all possible sets $B$.
        \end{ex}

        \begin{aside}
          \begin{align*}
            \bigcap \left\{ [0, r] : r > 0 \right\} &= \set{0} \\
            \bigcap \left\{ (0, r) : r > 0 \right\} &= \emptyset \\
            \bigcup \left\{ \set{1, \ldots, n} : n \in \mathbb{N} \right\} &= \mathbb{N} \\
            \bigcap \left\{ \set{1, \ldots, n} : n \in \mathbb{N} \right\} &= \set{1}
          \end{align*}
        \end{aside}

        The equivalence of the two definitions of $I(A, P)$ is not immediately clear. We claim that $I(A, P)$ is closed under the operations of $P$, and $A \subseteq I(A, P)$.

        \begin{proof}
          Pick any $x, y \in I(A, P)$ and operation $f \in P$. Therefore, for every $B$ which contains $A$ and is closed under $P$, $x, y \in B$.
          \\ \\
          Since every such $B$ is closed under $P$, $f(x, y) \in B$ for every such $B$. Therefore, $f(x, y) \in \cap \set{ B : A \subseteq B \text{ and } B \text{ is closed under } P}$.
          \\ \\
          \underline{Exercise}: prove that $A \subseteq I(A, P)$.
        \end{proof}

        \begin{corollary}[Proof by Structural Induction Theory]
          Any $B \subseteq X$ which contains $A$ and is closed under $P$ is a superset of $I(A, P)$. Namely, $I(A, P) \subseteq B$.
        \end{corollary}

        We could rephrase this corollary as follows: given any set $B$, if you want to prove that $I(A, P) \subseteq B$, it suffices to show:
        \begin{itemize}
          \item $A \subseteq B$, and
          \item $B$ is closed under $P$.
        \end{itemize}

      \subsubsection{Mathematical Induction}
        Let's say we want to show that for every $n$, $n^2 + n$ is even. A proof by usual (mathematical) induction requires us to show two things:
        \begin{itemize}
          \item Prove the claim holds for $n = 1$. This is equivalent to showing that $A \subseteq B$.
          \item Prove that if the claim holds for $n$, it also holds for $n + 1$. This is equivalent to showing that $B$ is closed under $P$.
        \end{itemize}

        How is this the same as structural induction? Usual induction shows that $\mathbb{N} = I(A, P)$, where $A = \set{1}$ and $P = \set{ \text{``+1''}}$. That is, mathematical induction is just one specific case of structural induction.

      \subsubsection{Examples of Proofs by Structural Induction}
        \begin{ex}
          Imagine we have three paper cups placed on a desk, with two of them facing upwards and one upside down. You must flip exactly two cups at a time. We want to get all of the cups facing upwards. Can we prove that this goal is not achievable?
          \\ \\
          Let $X$ be the set of all possible configurations of the cups. Let our core set $A$ contain only the state ``[up] [down] [up]''. Let our set of operations $P$ contain:
          \begin{itemize}
            \item $P_1$: flip the two rightmost cups.
            \item $P_2$: flip the two leftmost cups.
            \item $P_3$: flip the rightmost and leftmost cups.
          \end{itemize}

          We claim that the state ``[up] [up] [up]'' $\not \in I(A, P)$.
          \begin{proof}
            Let $B$ be the set of all configurations with an even number of ``up'' cups.
            \\ \\
            We want to show that $I(A, P) \subseteq B$. This would show that ``[up] [up] [up]'' is not attainable because it is not a member of $B$. We will use structural induction to show this.
            \\ \\
            \textbf{Induction base}: $A \subseteq B$.
            \\ \\
            \textbf{Induction step}: $B$ is closed under $P$. Namely, if $c \in B$, then $P_1(c), P_2(c)$, and $P_3(c) \in B$. There are three cases:
            \begin{itemize}
              \item We flip one ``up'' cup and one ``down'' cup. This is a difference of +0 ``up'' cups.
              \item We flip two ``up'' cups. This causes us to remove two (which is even) from an already even number, which gives us another even number of ``up'' cups.
              \item We flip two ``down'' cups. This causes us to add two (which is even) to an already even number, which gives us another even number of ``up'' cups.
            \end{itemize}

            Note that these three cases show us that no matter which operation we use, it is not possible to obtain all three ``up'' cups, since the number of ``up'' cups must always be even.
          \end{proof}
        \end{ex}

        \begin{ex}
          Recall in \ref{strConcatenationExample} we defined $X = \set{a, b}^\star$, $A = \set{a}$, and $P = $ \bigg\{ \AxiomC{X} \UnaryInfC{aX} \DisplayProof, \AxiomC{X} \UnaryInfC{Xa} \DisplayProof, \AxiomC{X} \AxiomC{Y} \BinaryInfC{bXY} \DisplayProof, \AxiomC{X} \AxiomC{Y} \BinaryInfC{XbY} \DisplayProof, \AxiomC{X} \AxiomC{Y} \BinaryInfC{XYb} \DisplayProof \bigg\}.
          \\ \\
          For this example, let $\#_a(x)$ denote the number of ``a''s in the string $x$ and similarly, let $\#_b(x)$ denote the number of ``b''s in the string $x$.
          \\ \\
          \textbf{Claim}: every member of $I(A, P)$ has more ``a''s than ``b''s. To put this more simply, let $B = \set{ x : \#_a(x) > \#_b(x) }$. We wish to show that $I(A, P) \subseteq B$.
          \begin{proof}
            We will prove this claim by structural induction.
            \\ \\
            \textbf{Induction base}: $A \subseteq B$. It is trivial to see that the string ``a'' has more ``a''s than ``b''s (since it has no ``b''s).
            \\ \\
            \textbf{Induction step}: if $x, y \in B$, then we must show that $P_1(x) \in B, P_2(x) \in B, P_3(x, y) \in B, P_4(x,y) \in B$, and $P_5(x, y) \in B$.
            \\ \\
            It is clear that for $P_1$ and $P_2$, one ``a'' is being added to a string that must already include more ``a''s than ``b''s, so this trivially holds true.
            \\ \\
            The case of $P_3$ isn't as immediately obvious. Assume $\#_a(x) > \#_b(x)$, and $\#_a(y) > \#_b(y)$. So, $\#_a(x) \ge \#_b(x) + 1$ and $\#a(y) \ge \#_b(y) + 1$. This gives us $\#_a(xy) \ge \#_b(xy) + 2$.
            \\ \\
            Now, note that $\#_a(bxy) = \#_a(xy)$, and $\#_b(bxy) = \#_b(xy) + 1$. We can conclude that $\#_a(bxy) \ge \#_b(bxy) + 1 > \#_b(bxy)$.
            \\ \\
            A similar argument applies for $P_4$ and $P_5$.
          \end{proof}
        \end{ex}

      \subsubsection{Regular Languages}
        For us, tasks will always be decision problems. That is, given $L \subseteq \set{a, b}^\star$ and input $x \in \set{a, b}^\star$, decide if $x \in L$.
        \\ \\
        The set of all possible tasks is $\set{L : L \subseteq \set{a, b}^\star}$. This set is an infinite set of finite strings.
        \\ \\
        Recall that not all infinite sets have the same size. The set of all finite strings $\set{a, b}^\star$ is a ``small'' infinite set, meaning it is \textbf{countable}. However, the set of all subsets of $\set{a, b}^\star$ is a ``large'' infinite set, which is \textbf{uncountable}.
        \\ \\
        The set of all possible computer programs is a subset of $\set{0, 1}^\star$. Note that since every program is a \emph{finite} string, the set of all possible programs is countable.
        \\ \\
        Notice that we have an uncountable number of tasks, but a countable number of programs. This implies that there are tasks for which a language does not exist. That is, there are some tasks that cannot be carried out by programs.
\end{document}
